\documentclass[12pt,a4paper]{report}

% ===== Packages =====
\usepackage[utf8]{inputenc}
\usepackage[T1]{fontenc}
\usepackage[french]{babel}
\usepackage{csquotes} % recommandé avec biblatex
\usepackage{graphicx}
\usepackage{geometry}
\usepackage{titlesec}
\usepackage{fancyhdr}
\usepackage{setspace}
\usepackage{float}
\usepackage{caption}
\usepackage{appendix}
\usepackage{hyperref}
\usepackage{tocloft}
\usepackage[table]{xcolor}
\usepackage{array}
\usepackage{etoolbox} % utile pour certains ajustements
\usepackage{placeins}
% Bibliographie
\usepackage{enumitem}
\usepackage[backend=biber,style=numeric,sorting=none]{biblatex}
\addbibresource{webliography.bib}
 
\usepackage{subcaption}
% Glossaire / acronymes
\usepackage[acronym]{glossaries}
\makeglossaries

% ===== Marges =====
\geometry{
    a4paper,
    left=2.5cm,
    right=2.5cm,
    top=2.5cm,
    bottom=2.5cm,
    headheight=14pt, % adapté à fancyhdr
    headsep=0.6cm,
    footskip=1.2cm
}
\usepackage[a4paper,
            left=3cm,
            right=2.5cm,
            top=2.5cm,
            bottom=2.5cm,
            headheight=14pt,
            headsep=0.8cm,
            footskip=1.2cm]{geometry}

% ===== Interligne =====
\onehalfspacing

% ===== En-têtes / pieds =====
\pagestyle{fancy}
\fancyhf{}
\fancyhead[L]{\small\scshape\leftmark} % plus petit et petites majuscules
\fancyhead[R]{\thepage}
\renewcommand{\headrulewidth}{0.4pt}
\renewcommand{\footrulewidth}{0pt}

% ===== Titres =====
% Chapitre
\titleformat{\chapter}[display]
  {\normalfont\huge\bfseries\centering}
  {\chaptertitlename\ \thechapter}
  {20pt}
  {\Huge}
\titlespacing*{\chapter}{0pt}{-10pt}{30pt} % espacement avant/après

% Section
\titleformat{\section}
  {\normalfont\Large\bfseries}
  {\thesection}{1em}{}
\titlespacing*{\section}{0pt}{3ex plus .5ex minus .2ex}{2ex plus .2ex}

% Sous-section
\titleformat{\subsection}
  {\normalfont\large\bfseries}
  {\thesubsection}{1em}{}
\titlespacing*{\subsection}{0pt}{2.5ex plus .5ex minus .2ex}{1.5ex plus .2ex}

% ===== Légendes =====
\captionsetup{
    font=small,
    labelfont=bf,
    labelsep=period
}
\renewcommand{\figurename}{Figure}
\renewcommand{\tablename}{Tableau}

% ===== Table des matières =====
\setcounter{tocdepth}{3}
\setcounter{secnumdepth}{3}
\renewcommand{\cftchapleader}{\cftdotfill{\cftdotsep}}
\renewcommand{\cftsecleader}{\cftdotfill{\cftdotsep}}

% ===== Liens =====
\hypersetup{
    colorlinks=true,
    linkcolor=black,
    filecolor=black,      
    urlcolor=blue,
    citecolor=black,
    bookmarks=true,
    bookmarksopen=true,
    pdftitle={Rapport PFA - AutoStory - Générateur intelligent de contenus publicitaires pour le secteur automobile},
    pdfauthor={Youssef MOUSTAID et Fayz OUSSAMA}
}

% ===== Titre =====
\title{
  Rapport de Projet de Fin d'Année \\[0.5em]
  \Large AutoStory - Générateur intelligent de contenus publicitaires pour le secteur automobile
}
\author{
  Youssef MOUSTAID \\[0.2em]
  Fayz OUSSAMA \\[0.5em]
  \small Filière : Ingénierie Logicielle et Intelligence Artificielle \\
  \small 9ème Semestre \\[0.5em]
  \small Encadré par : Nom de l'Encadrant \\
  \small Année Universitaire 2025-2026
}
\date{Juin 2026}

% ===== Document =====
\begin{document}

% Page de garde
\begin{titlepage}
    \centering
    \includegraphics[width=0.35\textwidth]{images/logos/emsi_logo.png}\par\vspace{1cm}
    {\scshape\LARGE École Marocaine des Sciences de l'Ingénieur \par}
    \vspace{0.8cm}
    {\scshape\Large EMSI \par}
    \vspace{0.5cm}
    {\scshape\large Filière : Ingénierie Logicielle et Intelligence Artificielle\par}
    {\scshape\large 9ème Semestre\par}
    \vspace{1.5cm}
    {\huge\bfseries Rapport de Projet de Fin d'Année\par}
    \vspace{0.5cm}
    {\Large\bfseries AutoStory\par}
    \vspace{0.3cm}
    {\large Générateur intelligent de contenus publicitaires\par}
    {\large pour le secteur automobile\par}
    \vspace{2cm}
    {\Large Réalisé par\par}
    \vspace{0.5cm}
    {\large 
        \begin{tabular}{c}
            Youssef MOUSTAID \\
            Fayz OUSSAMA
        \end{tabular}
    \par}
    \vfill
    Encadré par\par
    Prénom NOM \textsc{de l'Encadrant}
    \vfill
    {\large Année Universitaire 2025-2026\par}
    {\large Juin 2026\par}
\end{titlepage}

% Page de validation
\clearpage








% dedicases
\chapter*{Dédicaces}
\addcontentsline{toc}{chapter}{Dédicaces}

Je tiens à dédier ce travail à toutes les personnes qui ont marqué mon parcours et qui m’ont permis de mener à bien ce \textbf{Projet de Fin d’Année}.  

\vspace{1em}

\textbf{À mes chers parents},  
pour leur amour incommensurable, leur patience et leurs sacrifices. Leur soutien moral, matériel et affectif a été la clé de ma réussite. Je leur dois ce que je suis aujourd’hui et leur exprime ma gratitude éternelle.  

\vspace{1em}

\textbf{À ma famille},  
frères, sœurs et proches, pour leurs encouragements constants, leur compréhension et leurs prières qui m’ont donné la force de persévérer même dans les moments difficiles.  

\vspace{1em}

\textbf{À mes enseignants et encadrants},  
qui m’ont transmis le savoir, guidé dans ma formation et accompagné tout au long de ce projet. Leur disponibilité et leurs conseils avisés ont grandement contribué à l’aboutissement de ce travail.  

\vspace{1em}

\textbf{À mes camarades de promotion et amis},  
avec qui j’ai partagé des moments de travail, de réflexion et d’amitié. Leur collaboration et leur esprit d’équipe ont rendu ce parcours plus agréable et enrichissant.  

\vspace{1em}

\textbf{Enfin, à toutes les personnes}, proches ou lointaines, qui ont cru en moi et m’ont soutenu dans cette aventure académique et personnelle. Ce travail est aussi le leur.  

% Remerciements
\chapter*{Remerciements}
\addcontentsline{toc}{chapter}{Remerciements}
Nous tenons à exprimer notre profonde gratitude à toutes les personnes qui ont contribué de près ou de loin à la réalisation de ce \textbf{Projet de Fin d'Année}.  
\vspace{1em}

Tout d'abord, nous remercions l'\textbf{École Marocaine des Sciences de l'Ingénieur (EMSI)} pour la formation de qualité dont nous avons bénéficié et pour nous avoir offert l'opportunité de travailler sur un projet innovant dans le domaine de l'intelligence artificielle appliquée au marketing automobile.
\vspace{1em}

Nous souhaitons également remercier \textbf{notre encadrant académique} pour son suivi rigoureux, ses conseils avisés et ses recommandations constructives, qui nous ont permis d'améliorer la qualité de ce travail et de mener à bien ce projet.  
\vspace{1em}

Nos remerciements vont également à tous les enseignants de la filière \textbf{Ingénierie Logicielle et Intelligence Artificielle} qui nous ont transmis les connaissances théoriques et pratiques nécessaires à la réalisation de ce projet.
\vspace{1em}

Enfin, nous adressons un grand merci à nos familles et nos amis pour leur soutien moral et leur encouragement constant durant cette période exigeante.

% Résumé
\chapter*{Résumé}
\addcontentsline{toc}{chapter}{Résumé}
Ce rapport présente le \textbf{Projet de Fin d'Année (PFA)} intitulé \textit{AutoStory}, réalisé à l'\textbf{École Marocaine des Sciences de l'Ingénieur (EMSI)} dans le cadre de la filière \textbf{Ingénierie Logicielle et Intelligence Artificielle}. L'objectif principal de ce projet est le développement d'un \textbf{système automatisé de génération de vidéos publicitaires} pour le secteur automobile, exploitant les capacités des \textbf{Large Language Models (LLMs)} et des technologies de synthèse vocale.
\vspace{1em}

La solution développée se compose d'un \textbf{backend Python} utilisant \textbf{Google Gemini AI 2.5 Flash} pour la génération de scripts narratifs techniques, \textbf{gTTS} pour la synthèse vocale française, et \textbf{MoviePy} pour la composition vidéo automatisée. Un \textbf{frontend React/TypeScript} est en cours de développement pour offrir une interface utilisateur moderne et intuitive. Le système permet de générer des vidéos publicitaires personnalisées en environ 3 minutes, adaptées au style du véhicule (sportif, luxe, familial, écologique).
\vspace{1em}

Ce projet académique a permis d'acquérir une expérience enrichissante en matière d'\textbf{intelligence artificielle générative}, de \textbf{développement full-stack}, d'\textbf{architecture logicielle modulaire}, et de traitement automatisé du langage naturel, tout en développant une solution innovante pour le marketing automobile.
\vspace{1em}

\bigskip
\textbf{Mots-clés :} Intelligence Artificielle, LLM, Gemini AI, Python, React, Génération automatique de contenu, Marketing automobile, Text-to-Speech, Composition vidéo.

% TOC et listes
\clearpage
\tableofcontents

\clearpage
\listoffigures
\addcontentsline{toc}{chapter}{Liste des figures}

\clearpage
\listoftables
\addcontentsline{toc}{chapter}{Liste des tableaux}

\clearpage
\chapter*{Liste des sigles et abréviations}
\addcontentsline{toc}{chapter}{Liste des sigles et abréviations}
\printglossary[type=\acronymtype,title=Liste des sigles et abréviations]

% --- Contenu ---

\chapter{Introduction}
\section*{Introduction Générale}

Ce rapport de \textbf{Projet de Fin d'Année (PFA)} présente le développement du projet \textbf{AutoStory}, réalisé dans le cadre de la formation en \textbf{Ingénierie Logicielle et Intelligence Artificielle} à l'\textbf{École Marocaine des Sciences de l'Ingénieur (EMSI)}.

\subsection*{Contexte}

Le marketing digital et le secteur automobile connaissent actuellement une transformation profonde portée par l'essor des technologies d'\textbf{intelligence artificielle générative}. Les entreprises du secteur automobile ont un besoin croissant de \textbf{contenus publicitaires personnalisés et produits rapidement} pour promouvoir leurs véhicules sur les plateformes digitales. Cependant, la production vidéo traditionnelle reste \textbf{coûteuse, chronophage et nécessite des compétences spécialisées} (scénaristes, voix-off professionnelles, monteurs vidéo).

L'émergence des \textbf{Large Language Models (LLMs)} tels que GPT-4, Claude et Google Gemini ouvre de nouvelles perspectives pour l'automatisation de la création de contenu narratif. Parallèlement, les technologies de \textbf{synthèse vocale (Text-to-Speech)} et de \textbf{composition vidéo programmatique} permettent aujourd'hui de produire des supports publicitaires de qualité sans intervention humaine extensive.

\subsection*{Problématique}

Face à ce contexte, plusieurs questions se posent :

\begin{itemize}
    \item Comment exploiter les capacités des LLMs pour générer des scripts publicitaires techniques et cohérents pour des véhicules automobiles ?
    \item Comment automatiser l'ensemble de la chaîne de production vidéo, de la génération du texte à la composition finale ?
    \item Comment garantir la qualité narrative et la cohérence des contenus générés automatiquement ?
    \item Quelle architecture logicielle adopter pour assurer modularité, scalabilité et maintenabilité du système ?
\end{itemize}

\subsection*{Objectifs du projet}

Le projet \textbf{AutoStory} vise à répondre à ces problématiques en développant un système complet de génération automatisée de vidéos publicitaires pour le secteur automobile. Les objectifs spécifiques sont les suivants :

\begin{enumerate}
    \item Mettre en place un système de génération de scripts narratifs techniques utilisant \textbf{Google Gemini 2.5 Flash}
    \item Développer un module de synthèse vocale en français avec \textbf{gTTS (Google Text-to-Speech)}
    \item Implémenter un système de composition vidéo automatisé avec \textbf{MoviePy}
    \item Créer un module de classification automatique des styles de véhicules (sportif, luxe, familial, écologique)
    \item Concevoir une interface utilisateur moderne avec \textbf{React et TypeScript}
    \item Adopter une architecture logicielle \textbf{modulaire et scalable} facilitant l'évolution future du système
\end{enumerate}

\subsection*{Démarche et structure du rapport}

Ce travail s'inscrit dans une démarche pédagogique rigoureuse, combinant apprentissage théorique et application pratique. Il nous permet de mobiliser nos connaissances en \textbf{intelligence artificielle}, \textbf{développement logiciel}, \textbf{traitement du langage naturel} et \textbf{architecture applicative}.

Le présent rapport est structuré de la manière suivante :

\begin{itemize}
    \item Le \textbf{Chapitre 1} présente le cahier des charges du projet AutoStory
    \item Le \textbf{Chapitre 2} expose le contexte général, l'état de l'art des technologies IA génératives et le positionnement d'AutoStory
    \item Le \textbf{Chapitre 3} détaille l'analyse et la spécification des besoins fonctionnels et non-fonctionnels
    \item Le \textbf{Chapitre 4} décrit la conception de l'architecture et les choix techniques
    \item Le \textbf{Chapitre 5} présente la réalisation et la mise en œuvre du système
    \item Le \textbf{Chapitre 6} conclut par un bilan du projet et présente les perspectives d'évolution
\end{itemize}

\chapter{Contexte Général et Présentation du Projet MobilLoca}
\label{chap:contexte}

\section{Introduction}
\label{sec:intro-contexte}
Dans ce chapitre, nous présenterons le cadre général de notre projet de fin d'études réalisé au sein de l'entreprise \textbf{Marketing Confort}, un acteur dynamique spécialisé dans le marketing digital et le développement de solutions technologiques innovantes. Nous débuterons par une présentation détaillée de l'organisme d'accueil, de ses activités, de sa vision et de ses compétences techniques. Ensuite, nous introduirons le projet \textbf{MobilLoca}, en explicitant son contexte, sa problématique, ses objectifs stratégiques et fonctionnels, ainsi que son plan de réalisation. Ce chapitre vise à poser les bases contextuelles et organisationnelles nécessaires à la compréhension des enjeux et de la portée du projet.

\section{L'organisme d'accueil}
\label{sec:organisme-accueil}

\subsection{Présentation de l'organisation}
\label{subsec:presentation-organisation}

\textbf{Marketing Confort} est une agence de marketing digital et de développement web fondée en 2013 et basée à Fès, avec une présence également en France. Elle accompagne ses clients dans la conception et la réalisation de projets digitaux sur mesure, allant de la stratégie marketing à la production audiovisuelle, en passant par le développement d'applications web et mobiles. L'entreprise se distingue par son approche personnalisée et son souci constant d'innovation, visant à offrir à ses clients un avantage concurrentiel durable grâce à des solutions digitales performantes et adaptées.

\begin{figure}[H]
    \centering
    \begin{minipage}{0.3\textwidth}
        \centering
        \includegraphics[width=\linewidth]{images/logos/marketing comfort.png}
        \caption{Logo de Marketing Confort}
        \label{fig:logo_marketing_confort}
    \end{minipage}
    \hfill
    \begin{minipage}{0.65\textwidth}
        \subsection{Mission et vision}
        \label{subsec:mission-vision}
        La mission de \textbf{Marketing Confort} est de rendre le marketing digital accessible et performant pour tous types d'entreprises, quelle que soit leur taille ou leur secteur d'activité. 
        Sa vision est de devenir un leader mondial dans le domaine du marketing numérique et de la production audiovisuelle, en établissant de nouveaux standards d'excellence et d'innovation.
    \end{minipage}
\end{figure}

\subsection{Services proposés}
\label{subsec:services-proposes}

L'agence propose une gamme étendue de services, incluant :

\begin{itemize}
    \item \textbf{Développement Web \& Mobile} : Conception et développement d'applications web et mobiles sur mesure, optimisées pour l'expérience utilisateur et le référencement.
    \item \textbf{Marketing Digital} : Stratégies de communication, campagnes publicitaires (Google Ads, Facebook Ads), gestion des réseaux sociaux et optimisation \gls{seo}.
    \item \textbf{Production Audiovisuelle} : Création de contenus vidéo promotionnels, spots publicitaires et supports visuels pour renforcer l'image de marque.
\end{itemize}

\begin{table}[H]
\centering
\renewcommand{\arraystretch}{1.5}
\caption{Fiche d'identité de Marketing Confort}
\label{tab:fiche_identite_mc}
\begin{tabular}{|>{\bfseries}p{0.3\textwidth}|p{0.6\textwidth}|}
\hline
\rowcolor{blue!25} \textbf{Information} & \textbf{Détail} \\
\hline
Nom de la société & Marketing Confort \\
\hline
ICE & 002720451000006 \\
\hline
Forme juridique & Société à Responsabilité Limitée \\
\hline
Date de création & 2013 \\
\hline
Siège Social & N°22 Lotissement Idrissi 3ème étage Atlas, Fès, Maroc \\
\hline
Coordonnées & +212 612 14 06 66 / +33760-297926 \\
 & contact@marketingconfort.com \\
\hline
Secteur d'activité & Services d'information commerciale \\
\hline
Nombre d'employés & 11 - 50 Employés \\
\hline
Capital & 100 000 DH \\
\hline
\end{tabular}
\end{table}
\cite{charika2025marketing}


\subsection{Partenaires stratégiques}
\label{subsec:partenaires-strategiques}

Marketing Confort collabore avec plusieurs partenaires prestigieux, tels que :

\begin{itemize}
    \item AMDL (Agence Marocaine de Développement de la Logistique)
    \item Eau Thermale Avène
    \item TotalEnergies
    \item Burger King
    \item Bridges to the Future
    \item Minéraux Maroc
    \item Le Phare Pressing
    \item iCi (La Réussite)
    \item Isyane (Senteurs et Saveurs)
\end{itemize}

Ces collaborations enrichissent son expertise et lui permettent d'intervenir dans des secteurs variés.

\subsection{Structure organisationnelle}
\label{subsec:structure-organisationnelle}

L'entreprise est dirigée par un CEO et un CTO, et s'appuie sur des équipes spécialisées en développement IT, marketing, production audiovisuelle, ainsi que sur un réseau de freelances et de partenaires.

\begin{figure}[H]
    \centering
    \includegraphics[width=0.7\textwidth]{images/chapitre 2 - contexte/organigrame.png}
    \caption{Organigramme de Marketing Confort}
    \label{fig:organigramme_mc}
\end{figure}

\subsection{Compétences techniques}
\label{subsec:competences-techniques}

Marketing Confort dispose de compétences techniques diversifiées, couvrant :

\begin{itemize}
    \item Développement Web \& Mobile
    \item Cloud Computing
    \item Marketing Digital \& Production Audiovisuelle
    \item Coaching Agile \& DevOps
    \item Intelligence Artificielle
\end{itemize}

\begin{figure}[H]
    \centering
    \includegraphics[width=0.8\textwidth]{images/chapitre 2 - contexte/competences.png}
    \caption{Compétences de l'entreprise}
    \label{fig:competences_mc}
\end{figure}

\section{Présentation du projet MobilLoca}
\label{sec:presentation-projet}

\subsection{Contexte du projet}
\label{subsec:contexte-projet}

Le projet \textbf{MobilLoca} a été initié à la demande d'\textbf{Adanev}, une entreprise française spécialisée dans le transport de personnes à mobilité réduite. Bien qu'Adanev dispose déjà d'une application de transport avec chauffeur (\textit{Adanev Cab}), elle souhaitait étendre son offre en proposant un service de location de véhicules sans chauffeur, avec une gestion centralisée et numérique de l'ensemble du processus.

\subsection{Problématique}
\label{subsec:problematique}

La gestion manuelle des réservations, de la flotte, des documents administratifs et des paiements entraînait plusieurs limitations :

\begin{itemize}
    \item Manque de visibilité sur les performances globales
    \item Processus longs et sujets aux erreurs
    \item Expérience client peu fluide et manque d'autonomie
    \item Difficulté à suivre les maintenances et les documents expirés
\end{itemize}

\subsection{Objectifs du projet}
\label{subsec:objectifs-projet}

\textbf{MobilLoca} vise à :

\begin{itemize}
    \item \textbf{Automatiser} la gestion des réservations, contrats, paiements et documents
    \item \textbf{Centraliser} la gestion de la flotte et des agences
    \item \textbf{Offrir une expérience client intuitive} via une application mobile
    \item \textbf{Fournir des outils d'aide à la décision} via des tableaux de bord et rapports analytiques
    \item \textbf{Garantir une solution scalable et sécurisée}
\end{itemize}

\subsection{Solution proposée}
\label{subsec:solution-proposee}

La solution retenue est une plateforme digitale complète comprenant :

\begin{itemize}
    \item Un \textbf{Back-Office Web} pour les agents : gestion des véhicules, réservations, documents, paiements, statistiques.
    \item Une \textbf{Application Mobile} pour les clients : consultation du catalogue, réservation en ligne, paiement sécurisé, gestion des documents et historique.
\end{itemize}

\section{Planification du projet}
\label{sec:planification-projet}

\subsection{Méthodologie de gestion de projet}
\label{subsec:methodologie-gestion}

Une approche \textbf{Agile Scrum} a été adoptée pour garantir flexibilité, réactivité et livraison incrémentale. Le projet a été découpé en sprints de 2 à 3 semaines, avec des revues régulières et des ajustements continus.

\subsection{Phases principales}
\label{subsec:phases-principales}

\begin{enumerate}
    \item \textbf{Spécifications et conception} (Février 03 - Mai 01 2025)
    \item \textbf{Développement Front-End Web} (Mars 03 - Mai 16 2025)
    \item \textbf{Développement Front-End Mobile} (Février 16 - Mai 16 2025)
    \item \textbf{Structuration} (Avril 17 - Mai 17 2025)
    \item \textbf{Initiation des micro-services Back-End} (Mai 19 - Juin 02 2025)
    \item \textbf{Développement des micro-services Back-End} (Mai 19 - Septembre 02 2025)
    \item \textbf{Intégration des micro-service dans les applications} (Juilled 03 - Septembre 02 2025)
\end{enumerate}

\begin{figure}[H]
    \centering
    \includegraphics[width=\textwidth]{images/chapitre 2 - contexte/gant.png}
    \caption{Diagramme de Gantt du projet MobilLoca}
    \label{fig:gantt_mobilloca}
\end{figure}
\begin{figure}[H]
    \centering
    \includegraphics[width=\textwidth]{images/chapitre 2 - contexte/mobiloca_timeline.png}
    \caption{Diagramme de Gantt du projet MobilLoca - Derniers phases }
    \label{fig:gantt_mobilloca}
\end{figure}

\subsection{Outils de collaboration}
\subsubsection{Les outils :}
\label{subsec:outils-collaboration}

Pour garantir une communication efficace et une gestion de projet optimale, nous avons utilisé plusieurs outils collaboratifs adaptés à nos besoins. Chaque outil joue un rôle spécifique dans le processus de développement, permettant à l'équipe de rester synchronisée et productive.

\vspace{0.5em} % petit espace avant la liste

% Jira
\noindent
\begin{minipage}[H]{0.1\textwidth}
    \includegraphics[width=\linewidth]{images/logos/jira.png}
\end{minipage}%
\hfill
\begin{minipage}[H]{0.85\textwidth}
    \textbf{Jira} : Outil de gestion des tâches et des sprints, Jira nous a permis de planifier, suivre et prioriser les activités de développement selon la méthodologie Agile Scrum. Chaque tâche, user story ou bug était clairement assigné, ce qui garantissait transparence et traçabilité tout au long du projet.
\end{minipage}

\vspace{0.5em} % espace entre les outils

% GitLab
\noindent
\begin{minipage}[H]{0.1\textwidth}
    \includegraphics[width=\linewidth]{images/logos/gitlab.png}
\end{minipage}%
\hfill
\begin{minipage}[H]{0.85\textwidth}
    \textbf{GitLab} : Nous avons utilisé GitLab pour le versioning et l'intégration continue. Chaque membre de l'équipe pouvait gérer son code via des branches, des commits et des merge requests. Les pipelines CI/CD automatisés ont permis de tester et déployer les applications sans erreurs, améliorant la qualité globale du projet.
\end{minipage}

\vspace{0.5em}

% Discord
\noindent
\begin{minipage}[H]{0.1\textwidth}
    \includegraphics[width=\linewidth]{images/logos/discord.png}
\end{minipage}%
\hfill
\begin{minipage}[H]{0.85\textwidth}
    \textbf{Discord} : Pour la communication en temps réel, nous avons utilisé Discord, ce qui a facilité les discussions instantanées, les réunions d'équipe virtuelles et le partage de fichiers. L'outil a contribué à maintenir un flux de communication constant entre tous les membres.
\end{minipage}

\vspace{0.5em}

% Titan-Mail
\noindent
\begin{minipage}[H]{0.1\textwidth}
    \includegraphics[width=\linewidth]{images/logos/titan.png}
\end{minipage}%
\hfill
\begin{minipage}[H]{0.85\textwidth}
    \textbf{Titan-Mail} : Titan-Mail a été utilisé pour la messagerie professionnelle, garantissant la gestion formelle des communications, le suivi des échanges et la réception des notifications importantes liées au projet.
\end{minipage}


\subsubsection{Extraits et exemples :}

% === Jira examples ===
\noindent
\textbf{Jira :} Exemples de gestion des sprints et des tâches.

\begin{figure}[H]
    \centering
    \includegraphics[width=0.9\textwidth]{images/chapitre 2 - contexte/board.png}
    \caption{Exemple d’un sprint board dans Jira}
    \label{fig:jira-sprint-board}
\end{figure}

\vspace{1em}

% === GitLab examples ===
\noindent
\textbf{GitLab :} Exemples de gestion du code, merge requests et pipelines CI/CD.

\begin{figure}[H]
    \centering
    \includegraphics[width=0.85\textwidth]{images/chapitre 2 - contexte/merge-request.png}
    \caption{Exemple de merge request dans GitLab}
    \label{fig:gitlab-mr}
\end{figure}

\begin{figure}[H]
    \centering
    \includegraphics[width=0.85\textwidth]{images/chapitre 2 - contexte/git-lab-modules.png}
    \caption{Modules de projet et structure dans GitLab}
    \label{fig:gitlab-modules}
\end{figure}

\vspace{1em}

% === Discord examples ===
\noindent
\textbf{Discord :} Communication en temps réel et suivi des discussions de projet.

\begin{figure}[H]
    \centering
    \includegraphics[width=0.85\textwidth]{images/chapitre 2 - contexte/asking-devops-engineer-to-validate-mr.png}
    \caption{Exemple de demandes et discussions dans Discord}
    \label{fig:discord-requests}
\end{figure}

\vspace{1em}


\section{Conclusion}
\label{sec:conclusion-contexte}

Ce chapitre a permis de situer le projet \textbf{MobilLoca} dans son contexte organisationnel et stratégique. Nous avons présenté l'entreprise d'accueil, ses services, sa vision, ainsi que les enjeux et objectifs du projet. La méthodologie Agile et les outils de gestion mis en place assurent un cadre de travail structuré et réactif. Le chapitre suivant détaillera l'analyse des besoins et la conception technique de l'application.
\chapter{Analyse et Spécification des Besoins}

\section{Introduction}

Dans ce chapitre, nous présenterons l'analyse détaillée des besoins du projet AutoStory. Cette phase constitue une étape fondamentale dans le développement du système, car elle permet d'identifier et de formaliser les exigences fonctionnelles et non-fonctionnelles. Nous établirons également les spécifications techniques qui guideront la conception et l'implémentation de la solution.

\section{Besoins fonctionnels}
\label{sec:besoins-fonctionnels}

Les besoins fonctionnels décrivent les fonctionnalités que le système AutoStory doit offrir aux utilisateurs.

\subsection{Génération de scripts narratifs}
\label{subsec:generation-scripts}

\textbf{BF1 : Génération automatique de scripts publicitaires}
\begin{itemize}
    \item Le système doit générer des scripts narratifs techniques et cohérents pour des véhicules automobiles
    \item Le script doit inclure des informations sur les caractéristiques techniques, le design et les innovations
    \item La longueur du script doit être adaptée à une vidéo de 30-60 secondes
    \item Le ton et le style doivent s'adapter au type de véhicule (sportif, luxe, familial, écologique)
\end{itemize}

\textbf{BF2 : Utilisation d'un Large Language Model}
\begin{itemize}
    \item Le système doit intégrer Google Gemini 2.5 Flash pour la génération de textes
    \item Les prompts doivent être optimisés pour obtenir des résultats de qualité professionnelle
    \item Le système doit gérer les erreurs de communication avec l'API Gemini
\end{itemize}

\subsection{Classification des véhicules}
\label{subsec:classification-vehicules}

\textbf{BF3 : Détection automatique du style de véhicule}
\begin{itemize}
    \item Le système doit classifier automatiquement les véhicules en catégories (sportif, luxe, familial, écologique)
    \item La classification doit se baser sur les caractéristiques techniques et visuelles du véhicule
    \item Le système doit adapter le ton narratif en fonction de la catégorie détectée
\end{itemize}

\subsection{Synthèse vocale}
\label{subsec:synthese-vocale}

\textbf{BF4 : Conversion texte vers audio}
\begin{itemize}
    \item Le système doit convertir le script généré en audio avec une voix française naturelle
    \item La qualité audio doit être professionnelle (format MP3, bitrate adapté)
    \item La synthèse vocale doit gérer correctement la prononciation des termes techniques automobiles
\end{itemize}

\subsection{Composition vidéo}
\label{subsec:composition-video}

\textbf{BF5 : Génération automatique de vidéos}
\begin{itemize}
    \item Le système doit composer automatiquement une vidéo à partir d'images du véhicule
    \item La vidéo doit synchroniser les images avec la narration audio
    \item Le système doit ajouter des transitions fluides entre les images
    \item La vidéo finale doit être exportée en format MP4 haute définition (1080p minimum)
\end{itemize}

\textbf{BF6 : Personnalisation visuelle}
\begin{itemize}
    \item Le système doit permettre l'ajout de textes overlay (nom du véhicule, slogan)
    \item Les effets visuels doivent correspondre au style du véhicule
    \item Le système doit gérer différents formats d'images en entrée
\end{itemize}

\subsection{Interface utilisateur}
\label{subsec:interface-utilisateur}

\textbf{BF7 : Interface web intuitive}
\begin{itemize}
    \item L'utilisateur doit pouvoir uploader des images du véhicule
    \item L'utilisateur doit pouvoir saisir les caractéristiques techniques du véhicule
    \item Le système doit afficher l'état d'avancement de la génération (script, audio, vidéo)
    \item L'utilisateur doit pouvoir prévisualiser et télécharger la vidéo générée
\end{itemize}

\textbf{BF8 : Gestion de multiples générations}
\begin{itemize}
    \item L'utilisateur doit pouvoir consulter l'historique des vidéos générées
    \item Le système doit permettre de régénérer une vidéo avec des paramètres modifiés
    \item L'utilisateur doit pouvoir supprimer des vidéos de son historique
\end{itemize}

\section{Besoins non-fonctionnels}
\label{sec:besoins-non-fonctionnels}

Les besoins non-fonctionnels définissent les contraintes et les critères de qualité du système.

\subsection{Performance}
\label{subsec:performance}

\textbf{BNF1 : Temps de génération}
\begin{itemize}
    \item Le temps total de génération d'une vidéo ne doit pas excéder 5 minutes
    \item La génération du script doit prendre moins de 30 secondes
    \item La synthèse vocale doit prendre moins de 20 secondes
    \item La composition vidéo doit prendre moins de 3 minutes
\end{itemize}

\textbf{BNF2 : Optimisation des ressources}
\begin{itemize}
    \item Le système doit optimiser l'utilisation de la mémoire lors du traitement vidéo
    \item Les fichiers temporaires doivent être nettoyés après la génération
\end{itemize}

\subsection{Fiabilité}
\label{subsec:fiabilite}

\textbf{BNF3 : Gestion des erreurs}
\begin{itemize}
    \item Le système doit gérer gracieusement les échecs d'API (Gemini, gTTS)
    \item Les erreurs doivent être loggées avec des informations détaillées
    \item L'utilisateur doit recevoir des messages d'erreur clairs et actionnables
\end{itemize}

\textbf{BNF4 : Qualité des contenus}
\begin{itemize}
    \item Les scripts générés doivent être cohérents et sans erreurs grammaticales
    \item La qualité audio doit être professionnelle sans distorsions
    \item La qualité vidéo doit être HD (1920x1080 minimum)
\end{itemize}

\subsection{Maintenabilité}
\label{subsec:maintenabilite}

\textbf{BNF5 : Architecture modulaire}
\begin{itemize}
    \item Le code doit être organisé en modules indépendants (génération, synthèse, composition)
    \item Chaque module doit avoir une responsabilité unique et bien définie
    \item Le système doit permettre le remplacement facile d'un composant (ex: changer de LLM)
\end{itemize}

\textbf{BNF6 : Documentation}
\begin{itemize}
    \item Le code doit être commenté et documenté (docstrings Python)
    \item Un guide d'utilisation doit être fourni pour les utilisateurs
    \item Une documentation technique doit être disponible pour les développeurs
\end{itemize}

\subsection{Sécurité}
\label{subsec:securite}

\textbf{BNF7 : Protection des données}
\begin{itemize}
    \item Les clés API doivent être stockées de manière sécurisée (variables d'environnement)
    \item Les fichiers uploadés doivent être validés (type, taille, contenu)
    \item Les données utilisateur ne doivent pas être partagées avec des tiers
\end{itemize}

\subsection{Scalabilité}
\label{subsec:scalabilite}

\textbf{BNF8 : Évolutivité}
\begin{itemize}
    \item L'architecture doit permettre l'ajout de nouveaux styles de véhicules
    \item Le système doit supporter l'ajout de nouvelles langues pour la synthèse vocale
    \item Le système doit être prêt pour une migration vers le cloud si nécessaire
\end{itemize}

\subsection{Utilisabilité}
\label{subsec:utilisabilite}

\textbf{BNF9 : Expérience utilisateur}
\begin{itemize}
    \item L'interface doit être intuitive et ne nécessiter aucune formation
    \item Le système doit fournir des feedbacks visuels clairs sur l'état du traitement
    \item L'interface doit être responsive et accessible sur différents appareils
\end{itemize}

\section{Contraintes techniques}
\label{sec:contraintes-techniques}

\subsection{Contraintes technologiques}
\begin{itemize}
    \item \textbf{Backend :} Python 3.10+, FastAPI
    \item \textbf{Frontend :} React 18+, TypeScript 5+
    \item \textbf{LLM :} Google Gemini 2.5 Flash
    \item \textbf{TTS :} gTTS (Google Text-to-Speech)
    \item \textbf{Composition vidéo :} MoviePy, Pillow
    \item \textbf{Format vidéo :} MP4 (H.264)
\end{itemize}

\subsection{Contraintes d'environnement}
\begin{itemize}
    \item Le système doit fonctionner sur Linux et Windows
    \item Les dépendances doivent être gérées via pip/conda
    \item Le système doit supporter Python 3.10 minimum
\end{itemize}

\section{Spécifications techniques}
\label{sec:specifications-techniques}

\subsection{Architecture système}
\begin{itemize}
    \item \textbf{Backend API REST} avec FastAPI pour la communication frontend-backend
    \item \textbf{Module de génération} utilisant Google Gemini API
    \item \textbf{Module de synthèse vocale} utilisant gTTS
    \item \textbf{Module de composition vidéo} utilisant MoviePy
    \item \textbf{Module de classification} pour détecter le style de véhicule
    \item \textbf{Frontend React} avec interface utilisateur moderne
\end{itemize}

\subsection{Format des données}
\begin{itemize}
    \item \textbf{Images :} JPEG, PNG (max 10MB par image)
    \item \textbf{Audio :} MP3, 128 kbps minimum
    \item \textbf{Vidéo :} MP4 (H.264), 1920x1080, 30 fps
    \item \textbf{API :} JSON pour les échanges de données
\end{itemize}

\section{Conclusion}

Ce chapitre a permis d'identifier et de formaliser l'ensemble des besoins fonctionnels et non-fonctionnels du système AutoStory. Ces spécifications constituent la base pour la phase de conception et de développement. Le chapitre suivant présentera l'architecture technique et la modélisation de la solution.

\chapter{Analyse et Conception}

% Introduction non numérotée mais avec le même style
\section*{Introduction}
\addcontentsline{toc}{section}{Introduction}

Dans ce chapitre, nous allons présenter l'analyse et la conception de notre solution. Cette phase cruciale du développement nous permet de définir l'architecture globale du système, de modéliser les différents composants et leurs interactions, ainsi que de spécifier les fonctionnalités à travers différents diagrammes UML.

L'objectif de cette phase est de traduire les besoins identifiés lors de l'analyse fonctionnelle en une solution technique cohérente et réalisable. Nous commencerons par présenter l'architecture générale de la solution, puis nous détaillerons la conception à travers une approche de modélisation structurée.

% Section 1 - Architecture de la solution
\section{Architecture de la solution}

\subsection{Vue d'ensemble de l'architecture}

\begin{figure}[H]
\centering
\includegraphics[width=0.9\textwidth]{images/chapter-4/architecture.png}
\caption{Architecture du projet}
\label{fig:architecture}
\end{figure}

Pour architecturer notre projet, nous avons adopté une architecture orientée \textbf{microservices} afin de garantir l'évolutivité et la flexibilité nécessaires.  
Cette architecture se compose de plusieurs services indépendants, chacun dédié à un rôle spécifique, collaborant via une \textbf{passerelle (gateway)} centralisée qui orchestre les communications et assure une gestion sécurisée des requêtes.  

Un module commun (\textbf{common}) est également utilisé pour partager des fonctionnalités transversales, telles que les utilitaires ou les configurations standards, réduisant ainsi la redondance et favorisant la cohérence entre les services.  

Cette approche offre de nombreux avantages :  
\begin{itemize}
    \item une meilleure modularité,  
    \item une facilité de déploiement et de maintenance,  
    \item une agilité accrue pour répondre aux besoins évolutifs de l'entreprise.  
\end{itemize}

Chaque service peut être développé, testé, déployé et mis à jour indépendamment, favorisant une intégration continue et un déploiement continu alignés sur les pratiques \textbf{DevOps} adoptées par l'entreprise.

\subsection{Architecture globale du projet}

L'architecture de notre projet est conçue pour offrir une gestion modulaire et évolutive, intégrant à la fois des composants centraux et des microservices spécifiques.  

Le projet \textbf{mc-starter} est le cœur de notre architecture et fournit l'infrastructure de base pour le déploiement et l'exécution de l'application. Il est structuré en deux principaux packages :  

\begin{itemize}
    \item \textbf{mc-starter-app} : contient les fichiers de configuration nécessaires pour l'initialisation et la gestion des différents environnements (développement, test, production). Les fichiers comme \texttt{application.yml} définissent les paramètres globaux et permettent de configurer les services requis.  
    \item \textbf{mc-starter-core} : héberge la logique métier de l'application, avec des sous-packages spécialisés.  
\end{itemize}

\subsection{Approche de sécurité de l’application}

Pour renforcer la sécurité de l’application, nous avons choisi, comme pour tous les projets de \textbf{Marketing Confort}, d’utiliser \textbf{Keycloak}, une solution open source dédiée à la gestion des identités et des accès.  

Keycloak permet de centraliser l’authentification des utilisateurs et d’offrir des fonctionnalités essentielles telles que :  
\begin{itemize}
    \item l’authentification unique (SSO),  
    \item l’authentification multifactorielle (MFA),  
    \item une gestion fine des rôles, permissions et tokens d’accès.  
\end{itemize}

Cette intégration nous a permis de sécuriser efficacement l’accès aux différentes parties de l’application tout en simplifiant la connexion pour les utilisateurs.  
Keycloak s’avère être un atout précieux dans le cadre d’une \textbf{architecture microservices}, où la sécurité et la gestion des accès doivent être homogènes et centralisées.

\begin{figure}[h]
\centering
 \includegraphics[width=0.8\textwidth]{images/chapter-4/approcheSecurite.png}
\caption{Approche de sécurité de l’application}
\label{fig:approche_securite}
\end{figure}

% Section 2 - Conception et modélisation
\section{Conception et modélisation}

La phase de conception se base sur une approche de modélisation UML (Unified Modeling Language) qui nous permet de représenter de manière claire et précise les différents aspects de notre système. Cette modélisation comprend trois types de diagrammes principaux qui couvrent les aspects fonctionnels, structurels et comportementaux de la solution.

\subsection{Diagrammes de cas d'utilisation}

Les diagrammes de cas d'utilisation UML modélisent les interactions entre les acteurs (utilisateurs ou systèmes externes) et les fonctionnalités d'une application. Ils offrent une vision globale du comportement attendu du système, facilitant la communication entre parties prenantes et guidant le développement.

\subsubsection{Diagramme de cas d’utilisation du module gestion des réservations}

Le diagramme ci-dessous illustre les interactions du \textbf{Client} avec le système de réservation de véhicules, incluant :
\begin{itemize}
    \item Consulter la liste des véhicules disponibles
    \item Sélectionner un véhicule (avec options de filtrage)
    \item Effectuer une réservation (incluant choix des dates/lieu, options supplémentaires)
    \item Modifier/Annuler une réservation
    \item Consulter l'historique
\end{itemize}

Relations clés :
\begin{itemize}
    \item \textbf{Inclusion :} L'authentification est requise pour certaines actions
    \item \textbf{Extension :} Le choix d'options supplémentaires (GPS, assurance) étend le cas de base « Effectuer une réservation »
\end{itemize}

\begin{figure}[h]
\centering
 \includegraphics[width=1\textwidth]{images/chapter-4/usecaseReservation.png}
\caption{Diagramme de cas d'utilisation du module réservation}
\label{fig:use_case_reservation}
\end{figure}

\paragraph{Cas d'utilisation : Effectuer une réservation}\mbox{}\\

\begin{table}[H]
\centering
\begin{tabular}{|l|p{
10cm}|}
\hline
\textbf{Description} & Permet au client de réserver un véhicule après sélection. \\ \hline
\textbf{Acteurs} & Client \\ \hline
\textbf{Préconditions} & Client authentifié. \newline Véhicule disponible aux dates choisies. \\ \hline
\textbf{Scénario nominal} & 
1. Client sélectionne un véhicule. \newline
2. Choisit dates/lieu. \newline
3. Ajoute des options (GPS). \newline
4. Valide le paiement. \newline
5. Reçoit une confirmation. \\ \hline
\textbf{Scénario alternatif} & Dates indisponibles $\rightarrow$ le système propose des créneaux alternatifs. \\ \hline
\end{tabular}
\caption{Cas d'utilisation : Effectuer une réservation}
\label{tab:use_case_reservation1}
\end{table}

\paragraph{Cas d'utilisation : Modifier une réservation}\mbox{}\\

\begin{table}[H]
\centering
\begin{tabular}{|l|p{10cm}|}
\hline
\textbf{Description} & Permet de modifier les détails d'une réservation existante. \\ \hline
\textbf{Acteurs} & Client, Agent \\ \hline
\textbf{Préconditions} & Réservation existante, non débutée. \\ \hline
\textbf{Scénario nominal} & 
1. Client accède à « Mes réservations ». \newline
2. Modifie les dates. \newline
3. Système calcule le coût ajusté. \newline
4. Confirme les changements. \\ \hline
\textbf{Scénario alternatif} & Modification impossible (véhicule indisponible) $\rightarrow$ notification d'erreur. \\ \hline
\end{tabular}
\caption{Cas d'utilisation : Modifier une réservation}
\label{tab:use_case_reservation2}
\end{table}

\subsubsection{Diagramme de cas d’utilisation du module gestion des véhicules}

Réservé aux \textbf{Agents} et \textbf{Administrateurs}, ce diagramme couvre :
\begin{itemize}
    \item Ajouter / Modifier / Supprimer un véhicule
    \item Gérer la disponibilité (maintenance, calendrier)
    \item Assigner un véhicule à une agence
\end{itemize}

\begin{figure}[H]
\centering
 \includegraphics[width=0.8\textwidth]{images/chapter-4/useCaseVehicule.png}
\caption{Diagramme de cas d'utilisation du module véhicule}
\label{fig:use_case_vehicule}
\end{figure}

\paragraph{Cas d'utilisation : Ajouter un véhicule}\mbox{}\\

\begin{table}[H]
\centering
\begin{tabular}{|l|p{10cm}|}
\hline
\textbf{Description} & Ajoute un nouveau véhicule au catalogue. \\ \hline
\textbf{Acteurs} & Administrateur, Agent \\ \hline
\textbf{Préconditions} & Authentifié avec droits d'édition. \\ \hline
\textbf{Scénario nominal} & 
1. Remplit le formulaire (modèle, plaque, prix). \newline
2. Uploade des photos. \newline
3. Valide. \newline
4. Système ajoute le véhicule. \\ \hline
\textbf{Scénario alternatif} & Champs incomplets $\rightarrow$ formulaire bloqué jusqu'à correction. \\ \hline
\end{tabular}
\caption{Cas d'utilisation : Ajouter un véhicule}
\label{tab:use_case_vehicule1}
\end{table}

\paragraph{Cas d'utilisation : Assigner un véhicule à une agence}\mbox{}\\

\begin{table}[H]
\centering
\begin{tabular}{|l|p{10cm}|}
\hline
\textbf{Description} & Affecte un véhicule à une agence spécifique. \\ \hline
\textbf{Acteurs} & Administrateur \\ \hline
\textbf{Préconditions} & Véhicule et agence existants. \\ \hline
\textbf{Scénario nominal} & 
1. Sélectionne le véhicule. \newline
2. Choisit l'agence. \newline
3. Valide. \newline
4. Système met à jour la disponibilité. \\ \hline
\textbf{Scénario alternatif} & Agence déjà saturée $\rightarrow$ proposition d'une autre agence. \\ \hline
\end{tabular}
\caption{Cas d'utilisation : Assigner un véhicule à une agence}
\label{tab:use_case_vehicule2}
\end{table}

\noindent\textbf{Objectif :} Ces diagrammes servent de base pour les phases de développement et de tests, en clarifiant les attentes fonctionnelles.

\subsubsection{Diagramme de cas d’utilisation du module gestion des agences}

Réservé aux \textbf{Administrateurs}, ce diagramme couvre :
\begin{itemize}
    \item Consulter la liste des agences
    \item Consulter les détails d'une agence spécifique
    \item Suivre les performances
    \item Ajouter une nouvelle agence
    \item Modifier les détails d'une agence
    \item Supprimer une agence
\end{itemize}

\begin{figure}[H]
\centering
 \includegraphics[width=0.8\textwidth]{images/chapter-4/useCaseAgency.png}
\caption{Diagramme de cas d'utilisation du module agences}
\label{fig:use_case_agence}
\end{figure}

\paragraph{Cas d'utilisation : Ajouter une nouvelle agence}\mbox{}\\

\begin{table}[H]
\centering
\begin{tabular}{|l|p{10cm}|}
\hline
\textbf{Description} & Ajoute une nouvelle agence au système. \\ \hline
\textbf{Acteurs} & Administrateur \\ \hline
\textbf{Préconditions} & Authentifié avec droits d’administration. \\ \hline
\textbf{Scénario nominal} & 
1. Remplit le formulaire (nom, adresse, contact, etc.). \newline
2. Valide le formulaire. \newline
3. Le système enregistre l’agence dans la base de données. \\ \hline
\textbf{Scénario alternatif} & Champs manquants ou invalides $\rightarrow$ message d’erreur et correction demandée. \\ \hline
\end{tabular}
\caption{Cas d'utilisation : Ajouter une nouvelle agence}
\label{tab:use_case_agence1}
\end{table}

\paragraph{Cas d'utilisation : Consulter les détails d’une agence spécifique}\mbox{}\\

\begin{table}[H]
\centering
\begin{tabular}{|l|p{10cm}|}
\hline
\textbf{Description} & Affiche toutes les informations relatives à une agence donnée. \\ \hline
\textbf{Acteurs} & Administrateur \\ \hline
\textbf{Préconditions} & L’agence existe dans la base de données. \\ \hline
\textbf{Scénario nominal} & 
1. Sélectionne une agence dans la liste. \newline
2. Le système affiche les détails (nom, adresse, employés, véhicules liés, etc.). \\ \hline
\textbf{Scénario alternatif} & Agence introuvable $\rightarrow$ message d’erreur. \\ \hline
\end{tabular}
\caption{Cas d'utilisation : Consulter les détails d’une agence}
\label{tab:use_case_agence2}
\end{table}

\noindent\textbf{Objectif :} Ces diagrammes clarifient les fonctionnalités principales de la gestion des agences, et serviront de référence pour le développement et les tests.

\subsection{Diagrammes de séquence}

Les diagrammes de séquence illustrent les interactions entre les objets et services du système dans un ordre chronologique, permettant de représenter les scénarios fonctionnels de bout en bout.

\subsubsection{Diagramme de séquence – Processus d’authentification}

Le diagramme ci-dessous illustre le scénario d’authentification de l’utilisateur au sein de l’application MobiLoca.  
Il décrit les interactions entre le client, l’application mobile, la passerelle API, le système Keycloak (gestionnaire d’identité) et la base de données des utilisateurs.

\begin{figure}[H]
\centering
\includegraphics[width=0.9\textwidth]{images/chapter-4/seqAuth.png}
\caption{Diagramme de séquence – Processus d’authentification}
\label{fig:seq_auth}
\end{figure}

L’utilisateur commence par saisir ses identifiants sur l’application mobile, qui sont ensuite vérifiés par Keycloak via la passerelle API.  
Selon la validité des informations, un jeton d’accès sécurisé est délivré, ou un message d’erreur adapté est retourné en cas d’échec ou de compte désactivé.  
Ce mécanisme garantit une authentification fiable et sécurisée, essentielle pour protéger l’accès à l’application.

\subsubsection{Diagramme de séquence – Processus de réservation}

Une fois connecté, l’utilisateur peut accéder à la fonctionnalité principale de MobiLoca : la réservation d’un véhicule électrique.  
Le diagramme suivant illustre précisément ce processus clé, décrivant les interactions nécessaires pour sélectionner, réserver et confirmer une voiture via l’application.

\begin{figure}[H]
\centering
\includegraphics[width=0.9\textwidth]{images/chapter-4/seqReservation.png}
\caption{Diagramme de séquence – Processus de réservation}
\label{fig:seq_reservation}
\end{figure}

Ce diagramme illustre le processus complet de réservation d’un véhicule électrique par un utilisateur via l’application MobiLoca, en mettant en évidence les échanges entre l’utilisateur, l’application mobile et les différents microservices composant l’architecture backend :

\begin{enumerate}[label=\alph*.]
    \item \textbf{Consultation du catalogue} : l’utilisateur interroge le microservice Véhicule via la passerelle API pour récupérer la liste des véhicules disponibles.
    \item \textbf{Affichage des détails et options} : l’application récupère les détails du véhicule et ses packs/options (GPS, siège bébé, etc.) auprès du microservice Véhicule.
    \item \textbf{Récupération des informations utilisateur} : l’application interroge le microservice Utilisateur pour obtenir ou mettre à jour les informations personnelles nécessaires.
    \item \textbf{Ajout au panier} : l’utilisateur ajoute sa sélection, transmise au microservice Réservation pour un stockage temporaire.
    \item \textbf{Paiement sécurisé} : l’application initie une transaction via le microservice Paiement et l’API Stripe. Selon le résultat, la réservation est confirmée ou rejetée.
    \item \textbf{Confirmation de la réservation} : après paiement accepté, la réservation est créée dans le microservice Réservation, le statut du véhicule est mis à jour et un contrat est généré par le microservice Utilisateur, envoyé par email et application.
\end{enumerate}

Ce scénario garantit une expérience utilisateur fluide et sécurisée, en orchestrant plusieurs services pour gérer la disponibilité, la réservation et la facturation des véhicules, tout en assurant la cohérence et la réactivité du système.


\subsection{Diagrammes de classes}

Les diagrammes de classes représentent la structure statique du système en montrant les classes, leurs attributs, leurs méthodes et les relations entre elles.  
Ils permettent de visualiser l’architecture des microservices et leurs dépendances internes.

\subsubsection{Diagramme de classes de service gestion des utilisateurs}

Ce diagramme de classes représente la structure du microservice utilisateur, qui centralise la gestion des différents types d’utilisateurs du système, notamment les clients, les agents enregistrés et les agents non enregistrés, en héritant tous de la classe principale \textbf{User}.

\begin{figure}[H]
\centering
\includegraphics[width=1\textwidth]{images/chapter-4/classUser.png}
\caption{Diagramme de classes du module utilisateur}
\label{fig:class_user}
\end{figure}

La classe \textbf{User} regroupe les informations personnelles (nom, email, téléphone, adresse, etc.).  
Chaque utilisateur peut avoir plusieurs activités stockées dans \textbf{ActivityHistory}.  
Les agents peuvent être associés à un ou plusieurs \textbf{Role}, chacun définissant un ensemble de \textbf{Permission}.  
Chaque permission correspond à une action (lire, créer, mettre à jour, supprimer) sur un module spécifique du système.  
Ce modèle met en œuvre un contrôle d’accès granulaire basé sur les rôles, permettant de sécuriser et d’adapter dynamiquement les autorisations selon les profils des utilisateurs.

\subsubsection{Diagramme de classes de service gestion des réservations}

Ce diagramme de classes illustre l’architecture du microservice de réservation, responsable de la gestion des processus de réservation de véhicules.  
Il modélise deux types de réservation : \textbf{SingleReservation} (réservation simple) et \textbf{GlobalReservation} (réservation groupée), chacune associée à un client.

\begin{figure}[H]
\centering
\includegraphics[width=1\textwidth]{images/chapter-4/classReservation.png}
\caption{Diagramme de classes du module réservation}
\label{fig:class_reservation}
\end{figure}

Chaque réservation est liée à un \textbf{Contrat}, qui formalise la relation avec le client, et à un \textbf{Deposit}, qui peut générer des pénalités en cas de non-respect des conditions contractuelles.  
Le dépôt peut être payé par différents moyens (carte bancaire, PayPal, etc.) et suit un cycle de validation.

\subsubsection{Diagramme de classes de service gestion des agences}

Ce diagramme de classes représente la structure du microservice agence, qui centralise la gestion des agences de location de véhicules, incluant leurs informations opérationnelles, statistiques mensuelles et rapports de performance.

\begin{figure}[H]
\centering
\includegraphics[width=0.9\textwidth]{images/chapter-4/classAgency.png}
\caption{Diagramme de classes du module agence}
\label{fig:class_agency}
\end{figure}

La classe principale \textbf{Agency} regroupe les informations fondamentales d'une agence (nom, email, téléphone, date de création, description, image, statut). Chaque agence possède une \textbf{Adresse} complète et des \textbf{OpeningHours} définissant ses horaires d'ouverture.

Les agences génèrent des données mensuelles stockées dans \textbf{MonthlyData}, incluant le nombre de locations et le chiffre d'affaires. Le système permet également la création de \textbf{Report} (rapports) personnalisés pouvant inclure diverses statistiques (\textbf{StatisticType}) et être organisés en dossiers via \textbf{ReportFolder}.

Chaque agence est associée à un gestionnaire (\textbf{managerId}), une liste d'agents (\textbf{agentIds}), une flotte de véhicules (\textbf{vehicleIds}) et des équipements (\textbf{equipmentIds}). Ce modèle permet une gestion complète des performances opérationnelles et financières des agences, avec une capacité avancée de reporting et d'analyse des données.

\subsubsection{Diagramme de classes de service gestion des véhicules}

Ce diagramme de classes modélise la structure du microservice responsable de la gestion des véhicules, de leurs performances, entretiens, équipements et dépenses.

\begin{figure}[H]
\centering
\includegraphics[width=1\textwidth]{images/chapter-4/classVehicule.png}
\caption{Diagramme de classes du module véhicule}
\label{fig:class_vehicule}
\end{figure}

La classe centrale est \textbf{Vehicle}, qui regroupe les informations essentielles sur chaque véhicule (marque, modèle, année, type de carburant, nombre de places, etc.).  
Chaque véhicule peut être enrichi via l'entité \textbf{VehicleEquipment}, qui permet d’associer des équipements intégrés (\textbf{BuiltInEquipment}) ou additionnels (\textbf{AddOnEquipment}), catégorisés selon \textbf{EquipmentCategory}.  

Le suivi de l'activité des véhicules est organisé via la classe \textbf{CalendarActivities}, qui se décline en trois sous-types :  
\begin{itemize}
    \item \textbf{ReservationActivity} (liée à une réservation),
    \item \textbf{MaintenanceActivity} (planification d’entretien),
    \item \textbf{RegularActivity} (autres activités régulières).
\end{itemize}

La performance des véhicules est suivie via \textbf{VehiclePerformance}, incluant des mesures telles que le kilométrage, la consommation moyenne ou la vitesse maximale.  
La classe \textbf{Expense} enregistre toutes les dépenses associées aux véhicules (carburant, assurance, réparation, etc.), avec leur \textbf{ExpenseCategory} et \textbf{ExpenseStatus} (Validée ou Rejetée).

\subsubsection{Diagramme de classes du module Fleet}

Ce diagramme de classes illustre l'architecture du microservice \textbf{Fleet}, responsable de la gestion post-réservation et du suivi opérationnel des véhicules.  
Il modélise quatre entités principales : \textbf{Complaint} (réclamation), \textbf{FuelConsumption} (consommation de carburant), \textbf{Incident} et \textbf{MileageRecord} (enregistrement kilométrique).

\begin{figure}[H]
\centering
\includegraphics[width=1\textwidth]{images/chapter-4/classFleet.png}
\caption{Diagramme de classes du module Fleet}
\label{fig:class_fleet}
\end{figure}

Chaque entité est associée à des énumérations spécifiques définissant leurs états et types.  
Les réclamations (\textbf{Complaint}) peuvent être de différents types (facturation, dommages véhicule, qualité de service) et suivent un cycle de traitement avec des statuts allant de \texttt{OPEN} à \texttt{RESOLVED}.  
Les incidents couvrent les pannes, dommages, accidents et vols, avec un suivi de leur traitement.  
La consommation de carburant et les dépassements kilométriques sont liés au statut de paiement (\texttt{PENDING}/\texttt{BILLED}), permettant la facturation des frais supplémentaires.  

Ainsi, ce module assure le suivi complet des événements post-location et la gestion des coûts additionnels.

% ==========================
% Conclusion non numérotée mais avec le même style
\section*{Conclusion}
\addcontentsline{toc}{section}{Conclusion}

Ce chapitre a présenté l'analyse et la conception complète de notre solution. L'architecture proposée répond aux besoins identifiés tout en garantissant la scalabilité, la maintenabilité et la performance du système.

La modélisation UML réalisée à travers les diagrammes de cas d'utilisation, de classes et de séquence nous a permis de :
\begin{itemize}
    \item Clarifier les fonctionnalités attendues et les interactions avec les utilisateurs
    \item Structurer l'organisation des données et des traitements
    \item Définir précisément les flux d'exécution des principales fonctionnalités
\end{itemize}

Cette phase de conception constitue la base solide sur laquelle s'appuiera la phase d'implémentation présentée dans le chapitre suivant. Les choix architecturaux et les modèles définis guideront le développement et assureront la cohérence technique de la solution finale.

\chapter{Réalisation et Mise en Œuvre}

\section{Introduction}

\section{Outils de développement et technologies}
\subsection{Outils de développement}

\subsection{Technologies de développement}

\subsubsection{Backend}

\subsubsection{Frontend}

\subsection{Outils et services complémentaires}

\subsubsection{Sécurité et droits d’accès}

\subsubsection{Base de données}

\subsubsection{Tests et développement API}

\subsection{Mise en place du CI/CD et du versioning}


\section{Tests et Validation}

\subsection{Tests unitaires}


\subsection{Validation des données}

\subsubsection{Backend}

\subsubsection{Frontend}

\section{Présentation des Interfaces Utilisateur}

\subsection{Front Office (Application Mobile)}

\subsubsection{Interface de connexion (Login)}


\subsubsection{Page d’accueil et catalogue de véhicules}

\subsubsection{Interface de filtres avancés}


\subsubsection{Détails du véhicule sélectionné}


\subsubsection{Options supplémentaires}

\subsubsection{Résumé de la réservation}

\subsubsection{Panier de réservation}

\subsubsection{Paiement sécurisé}

\subsubsection{Confirmation et finalisation}


\subsubsection{ Signalement d’un incident}

\subsection{5.2 Back Office (Application Web)}

\subsubsection*{a. Accueil – Synthèse des tâches urgentes et alertes}


\subsubsection*{b. Suivi des réservations}

\subsubsection*{c. Tableaux de bord analytiques – Satisfaction \& Rentabilité}



\subsubsection*{d. Génération de rapports personnalisés}

\subsubsection*{e. Calendrier global des véhicules}

\subsubsection*{f. Audit des actions}


\subsubsection*{g. Suivi des documents et alertes d’expiration}

\subsubsection*{h. Paiement en agence et génération de facture}

\subsection*{Conclusion}



\section{Identification des acteurs}


\section{Cas d’utilisation principaux}

\subsection{Processus de réservation}


\section{Conclusion}


\chapter{Conclusion et perspectives}
\section*{Conclusion Générale}

Au terme de ce \textbf{Projet de Fin d’Année (PFA)}, réalisé au sein de \textbf{Marketing Confort} pour le client \textbf{Adanev}, nous avons conçu et développé \textbf{MobiLoca}, une plateforme numérique innovante destinée à moderniser la gestion de la location de véhicules adaptés. Cette solution s’articule autour d’un \textbf{back-office web} pour les agents et d’une \textbf{application mobile} pour les clients, permettant une gestion centralisée, une automatisation des processus et une expérience utilisateur optimisée.

Ce projet m’a offert une opportunité précieuse de mettre en pratique et de renforcer mes compétences en \textbf{développement logiciel}, en \textbf{architecture microservices} et en \textbf{intégration de solutions sécurisées}. Nous avons mobilisé des technologies modernes telles que \textbf{Java 17, Spring Boot, React Native et Next.js}, ainsi que des outils spécialisés comme \textbf{Stripe} pour les paiements et \textbf{Keycloak} pour la gestion des accès. L’adoption de la méthodologie agile \textbf{Scrum} (sprints de deux semaines, feedback utilisateur continu) a permis d’adapter les fonctionnalités aux besoins réels et de livrer une solution fiable et évolutive.

Ce travail m’a également permis de développer des compétences transversales essentielles :
\begin{itemize}
    \item maîtrise du \textbf{cycle de vie d’un projet logiciel} en contexte professionnel,
    \item conception et intégration d’une \textbf{architecture basée sur les microservices},
    \item \textbf{collaboration en équipe pluridisciplinaire} et gestion des interactions,
    \item communication efficace et \textbf{adaptation aux exigences d’un client réel}.
\end{itemize}

Par ailleurs, \textbf{MobiLoca} ouvre la voie à plusieurs perspectives d’évolution :
\begin{itemize}
    \item intégration de \textbf{modules d’intelligence artificielle} pour la prédiction de la demande et la maintenance proactive ;
    \item ajout d’un \textbf{chatbot intelligent} et de fonctionnalités d’\textbf{OCR} pour améliorer l’expérience client ;
    \item migration vers des applications mobiles natives (\textbf{Kotlin/Swift}) afin d’optimiser les performances ;
    \item adoption d’une \textbf{API GraphQL} pour renforcer l’interopérabilité avec les systèmes tiers ;
    \item déploiement \textbf{cloud} avec \textbf{Kubernetes}, garantissant scalabilité, haute disponibilité et fiabilité.
\end{itemize}

En conclusion, ce \textbf{PFA} a représenté une étape déterminante dans mon parcours académique et professionnel. Il démontre comment une architecture moderne, associée à des choix technologiques pertinents, peut transformer la gestion interne d’une entreprise tout en améliorant l’expérience de ses clients. Je suis fière d’avoir contribué à cette réalisation et convaincue que les compétences acquises dans ce cadre constitueront un socle solide pour ma future carrière dans le domaine de l’ingénierie logicielle.


% Bibliographie
\nocite{*}
\clearpage
\printbibliography[heading=bibintoc, title={Webliographie}]

% Annexes
\begin{appendices}

\chapter{Annexes}
\section*{Annexes}

\subsection*{Annexe 1 : Code source }

\begin{itemize}
    \item Code source complet de partie backend et frontend du projet AutoStory :
    \begin{itemize}
    \item Back-end : \url{https://github.com/JosephMoustaid/AutoStory-backend}
    \item Front-end : \url{https://github.com/JosephMoustaid/AutoStory-frontend}
    \end{itemize}
\end{itemize}


\end{appendices}

\end{document}
