
\chapter{Conclusion et perspectives}
\section*{Conclusion Générale}

Au terme de ce \textbf{Projet de Fin d’Année (PFA)}, réalisé au sein de \textbf{Marketing Confort} pour le client \textbf{Adanev}, nous avons conçu et développé \textbf{MobiLoca}, une plateforme numérique innovante destinée à moderniser la gestion de la location de véhicules adaptés. Cette solution s’articule autour d’un \textbf{back-office web} pour les agents et d’une \textbf{application mobile} pour les clients, permettant une gestion centralisée, une automatisation des processus et une expérience utilisateur optimisée.

Ce projet m’a offert une opportunité précieuse de mettre en pratique et de renforcer mes compétences en \textbf{développement logiciel}, en \textbf{architecture microservices} et en \textbf{intégration de solutions sécurisées}. Nous avons mobilisé des technologies modernes telles que \textbf{Java 17, Spring Boot, React Native et Next.js}, ainsi que des outils spécialisés comme \textbf{Stripe} pour les paiements et \textbf{Keycloak} pour la gestion des accès. L’adoption de la méthodologie agile \textbf{Scrum} (sprints de deux semaines, feedback utilisateur continu) a permis d’adapter les fonctionnalités aux besoins réels et de livrer une solution fiable et évolutive.

Ce travail m’a également permis de développer des compétences transversales essentielles :
\begin{itemize}
    \item maîtrise du \textbf{cycle de vie d’un projet logiciel} en contexte professionnel,
    \item conception et intégration d’une \textbf{architecture basée sur les microservices},
    \item \textbf{collaboration en équipe pluridisciplinaire} et gestion des interactions,
    \item communication efficace et \textbf{adaptation aux exigences d’un client réel}.
\end{itemize}

Par ailleurs, \textbf{MobiLoca} ouvre la voie à plusieurs perspectives d’évolution :
\begin{itemize}
    \item intégration de \textbf{modules d’intelligence artificielle} pour la prédiction de la demande et la maintenance proactive ;
    \item ajout d’un \textbf{chatbot intelligent} et de fonctionnalités d’\textbf{OCR} pour améliorer l’expérience client ;
    \item migration vers des applications mobiles natives (\textbf{Kotlin/Swift}) afin d’optimiser les performances ;
    \item adoption d’une \textbf{API GraphQL} pour renforcer l’interopérabilité avec les systèmes tiers ;
    \item déploiement \textbf{cloud} avec \textbf{Kubernetes}, garantissant scalabilité, haute disponibilité et fiabilité.
\end{itemize}

En conclusion, ce \textbf{PFA} a représenté une étape déterminante dans mon parcours académique et professionnel. Il démontre comment une architecture moderne, associée à des choix technologiques pertinents, peut transformer la gestion interne d’une entreprise tout en améliorant l’expérience de ses clients. Je suis fière d’avoir contribué à cette réalisation et convaincue que les compétences acquises dans ce cadre constitueront un socle solide pour ma future carrière dans le domaine de l’ingénierie logicielle.
