\chapter{Contexte Général du Projet AutoStory}
\label{chap:contexte}

\section{Introduction}
\label{sec:intro-contexte}

Dans ce chapitre, nous présentons le contexte général du projet \textbf{AutoStory}, en abordant les enjeux actuels du marketing automobile digital, l'état de l'art des technologies d'intelligence artificielle générative, ainsi que le positionnement de notre solution dans cet écosystème. Ce chapitre vise à fournir une compréhension approfondie des motivations techniques et commerciales qui ont guidé la conception de ce système de génération automatisée de vidéos publicitaires.

\section{Le marketing automobile à l'ère du digital}
\label{sec:marketing-auto}

\subsection{Évolution du secteur automobile}
\label{subsec:evolution-secteur}

Le secteur automobile connaît une transformation digitale majeure. Les concessionnaires et constructeurs automobiles investissent massivement dans la présence en ligne et le marketing digital pour atteindre leurs clients. La vidéo publicitaire est devenue un format privilégié pour promouvoir les véhicules sur les plateformes sociales (YouTube, Facebook, Instagram, TikTok) et les sites web commerciaux.

\subsection{Les défis de la production vidéo traditionnelle}
\label{subsec:defis-production}

La production de contenus vidéo publicitaires présente plusieurs contraintes majeures :

\begin{itemize}
    \item \textbf{Coûts élevés :} Une vidéo publicitaire professionnelle nécessite un budget significatif (équipe de tournage, matériel, post-production)
    \item \textbf{Délais de production :} Le processus complet peut prendre plusieurs semaines (écriture du script, enregistrement voix-off, montage)
    \item \textbf{Compétences spécialisées :} Rédacteurs, comédiens voix-off, monteurs vidéo professionnels
    \item \textbf{Scalabilité limitée :} Difficulté à produire des contenus personnalisés à grande échelle
\end{itemize}

\section{L'intelligence artificielle générative}
\label{sec:ia-generative}

\subsection{Les Large Language Models (LLMs)}
\label{subsec:llms}

Les \textbf{Large Language Models} représentent une avancée majeure en traitement du langage naturel. Des modèles comme GPT-4 (OpenAI), Claude (Anthropic), et Gemini (Google) sont capables de générer des textes cohérents, techniques et créatifs dans de multiples domaines.

\subsubsection{Google Gemini 2.5 Flash}

Pour AutoStory, nous avons sélectionné \textbf{Google Gemini 2.5 Flash} pour plusieurs raisons :

\begin{itemize}
    \item Excellente capacité de génération de textes techniques et narratifs
    \item Support multilingue de qualité (notamment le français)
    \item Vitesse de génération optimale (modèle "Flash")
    \item API stable et bien documentée
    \item Coût d'utilisation compétitif
\end{itemize}

\subsection{La synthèse vocale (Text-to-Speech)}
\label{subsec:tts}

Les technologies de synthèse vocale ont considérablement progressé ces dernières années. Les voix synthétiques modernes sont difficilement distinguables des voix humaines. Pour notre projet, nous utilisons \textbf{gTTS (Google Text-to-Speech)}, qui offre :

\begin{itemize}
    \item Une qualité vocale naturelle et fluide
    \item Un support excellent du français
    \item Une intégration simple via API Python
    \item Une fiabilité éprouvée
\end{itemize}

\subsection{La composition vidéo programmatique}
\label{subsec:composition-video}

La bibliothèque \textbf{MoviePy} permet la création et l'édition de vidéos de manière programmatique en Python. Elle offre :

\begin{itemize}
    \item Manipulation d'images et de clips vidéo
    \item Synchronisation audio-vidéo précise
    \item Ajout de textes, transitions et effets
    \item Exportation dans divers formats (MP4, AVI, etc.)
\end{itemize}

\section{État de l'art des solutions existantes}
\label{sec:etat-art}

\subsection{Solutions commerciales}
\label{subsec:solutions-commerciales}

Plusieurs solutions commerciales existent dans le domaine de la génération automatique de contenus :

\begin{table}[H]
\centering
\renewcommand{\arraystretch}{1.5}
\caption{Comparaison des solutions existantes}
\label{tab:solutions_existantes}
\begin{tabular}{|p{0.25\textwidth}|p{0.35\textwidth}|p{0.3\textwidth}|}
\hline
\rowcolor{blue!25} \textbf{Solution} & \textbf{Fonctionnalités} & \textbf{Limitations} \\
\hline
Synthesia & Génération de vidéos avec avatars IA & Coût élevé, peu de personnalisation \\
\hline
Pictory & Création vidéo à partir de textes & Orienté réseaux sociaux, templates limités \\
\hline
Lumen5 & Transformation d'articles en vidéos & Pas de focus automobile \\
\hline
\textbf{AutoStory} & Génération spécialisée secteur auto & Solution académique en développement \\
\hline
\end{tabular}
\end{table}

\subsection{Positionnement d'AutoStory}
\label{subsec:positionnement}

AutoStory se distingue par :

\begin{itemize}
    \item \textbf{Spécialisation automobile :} Scripts techniques adaptés au secteur
    \item \textbf{Classification automatique :} Détection du style de véhicule (sportif, luxe, familial, écologique)
    \item \textbf{Architecture modulaire :} Facilité d'évolution et d'adaptation
    \item \textbf{Open source :} Projet académique avec vocation éducative
    \item \textbf{Personnalisation avancée :} Adaptation du ton narratif au style du véhicule
\end{itemize}

\section{Objectifs et portée du projet}
\label{sec:objectifs}

\subsection{Objectifs principaux}
\label{subsec:objectifs-principaux}

Le projet AutoStory vise à :

\begin{enumerate}
    \item Développer un système end-to-end de génération de vidéos publicitaires automobiles
    \item Exploiter les capacités des LLMs pour créer des scripts narratifs de qualité
    \item Automatiser entièrement la chaîne de production (texte → voix → vidéo)
    \item Fournir une interface utilisateur intuitive et moderne
    \item Adopter une architecture logicielle évolutive et maintenable
\end{enumerate}

\subsection{Périmètre du projet}
\label{subsec:perimetre}

Le projet couvre :

\begin{itemize}
    \item \textbf{Backend Python} avec API REST
    \item Module de génération de scripts avec Google Gemini
    \item Module de synthèse vocale avec gTTS
    \item Module de composition vidéo avec MoviePy
    \item Classification automatique des styles de véhicules
    \item \textbf{Frontend React/TypeScript} (en développement)
\end{itemize}

\subsection{Valeur ajoutée}
\label{subsec:valeur-ajoutee}

AutoStory apporte une valeur ajoutée significative :

\begin{itemize}
    \item \textbf{Réduction des coûts :} Élimination des coûts de production traditionnelle
    \item \textbf{Gain de temps :} Génération en ~3 minutes vs plusieurs semaines
    \item \textbf{Scalabilité :} Production de multiples variantes sans effort additionnel
    \item \textbf{Personnalisation :} Adaptation automatique au style du véhicule
    \item \textbf{Accessibilité :} Démocratisation de la production vidéo professionnelle
\end{itemize}

\section{Conclusion}
\label{sec:conclusion-contexte}

Ce chapitre a présenté le contexte global du projet AutoStory, en mettant en évidence les enjeux du marketing automobile digital, l'état de l'art des technologies IA génératives, et le positionnement de notre solution. Dans le chapitre suivant, nous détaillerons l'analyse et la spécification des besoins fonctionnels et non-fonctionnels du système.