\chapter{Contexte Général et Présentation du Projet MobilLoca}
\label{chap:contexte}

\section{Introduction}
\label{sec:intro-contexte}
Dans ce chapitre, nous présenterons le cadre général de notre projet de fin d'études réalisé au sein de l'entreprise \textbf{Marketing Confort}, un acteur dynamique spécialisé dans le marketing digital et le développement de solutions technologiques innovantes. Nous débuterons par une présentation détaillée de l'organisme d'accueil, de ses activités, de sa vision et de ses compétences techniques. Ensuite, nous introduirons le projet \textbf{MobilLoca}, en explicitant son contexte, sa problématique, ses objectifs stratégiques et fonctionnels, ainsi que son plan de réalisation. Ce chapitre vise à poser les bases contextuelles et organisationnelles nécessaires à la compréhension des enjeux et de la portée du projet.

\section{L'organisme d'accueil}
\label{sec:organisme-accueil}

\subsection{Présentation de l'organisation}
\label{subsec:presentation-organisation}

\textbf{Marketing Confort} est une agence de marketing digital et de développement web fondée en 2013 et basée à Fès, avec une présence également en France. Elle accompagne ses clients dans la conception et la réalisation de projets digitaux sur mesure, allant de la stratégie marketing à la production audiovisuelle, en passant par le développement d'applications web et mobiles. L'entreprise se distingue par son approche personnalisée et son souci constant d'innovation, visant à offrir à ses clients un avantage concurrentiel durable grâce à des solutions digitales performantes et adaptées.

\begin{figure}[H]
    \centering
    \begin{minipage}{0.3\textwidth}
        \centering
        \includegraphics[width=\linewidth]{images/logos/marketing comfort.png}
        \caption{Logo de Marketing Confort}
        \label{fig:logo_marketing_confort}
    \end{minipage}
    \hfill
    \begin{minipage}{0.65\textwidth}
        \subsection{Mission et vision}
        \label{subsec:mission-vision}
        La mission de \textbf{Marketing Confort} est de rendre le marketing digital accessible et performant pour tous types d'entreprises, quelle que soit leur taille ou leur secteur d'activité. 
        Sa vision est de devenir un leader mondial dans le domaine du marketing numérique et de la production audiovisuelle, en établissant de nouveaux standards d'excellence et d'innovation.
    \end{minipage}
\end{figure}

\subsection{Services proposés}
\label{subsec:services-proposes}

L'agence propose une gamme étendue de services, incluant :

\begin{itemize}
    \item \textbf{Développement Web \& Mobile} : Conception et développement d'applications web et mobiles sur mesure, optimisées pour l'expérience utilisateur et le référencement.
    \item \textbf{Marketing Digital} : Stratégies de communication, campagnes publicitaires (Google Ads, Facebook Ads), gestion des réseaux sociaux et optimisation \gls{seo}.
    \item \textbf{Production Audiovisuelle} : Création de contenus vidéo promotionnels, spots publicitaires et supports visuels pour renforcer l'image de marque.
\end{itemize}

\begin{table}[H]
\centering
\renewcommand{\arraystretch}{1.5}
\caption{Fiche d'identité de Marketing Confort}
\label{tab:fiche_identite_mc}
\begin{tabular}{|>{\bfseries}p{0.3\textwidth}|p{0.6\textwidth}|}
\hline
\rowcolor{blue!25} \textbf{Information} & \textbf{Détail} \\
\hline
Nom de la société & Marketing Confort \\
\hline
ICE & 002720451000006 \\
\hline
Forme juridique & Société à Responsabilité Limitée \\
\hline
Date de création & 2013 \\
\hline
Siège Social & N°22 Lotissement Idrissi 3ème étage Atlas, Fès, Maroc \\
\hline
Coordonnées & +212 612 14 06 66 / +33760-297926 \\
 & contact@marketingconfort.com \\
\hline
Secteur d'activité & Services d'information commerciale \\
\hline
Nombre d'employés & 11 - 50 Employés \\
\hline
Capital & 100 000 DH \\
\hline
\end{tabular}
\end{table}
\cite{charika2025marketing}


\subsection{Partenaires stratégiques}
\label{subsec:partenaires-strategiques}

Marketing Confort collabore avec plusieurs partenaires prestigieux, tels que :

\begin{itemize}
    \item AMDL (Agence Marocaine de Développement de la Logistique)
    \item Eau Thermale Avène
    \item TotalEnergies
    \item Burger King
    \item Bridges to the Future
    \item Minéraux Maroc
    \item Le Phare Pressing
    \item iCi (La Réussite)
    \item Isyane (Senteurs et Saveurs)
\end{itemize}

Ces collaborations enrichissent son expertise et lui permettent d'intervenir dans des secteurs variés.

\subsection{Structure organisationnelle}
\label{subsec:structure-organisationnelle}

L'entreprise est dirigée par un CEO et un CTO, et s'appuie sur des équipes spécialisées en développement IT, marketing, production audiovisuelle, ainsi que sur un réseau de freelances et de partenaires.

\begin{figure}[H]
    \centering
    \includegraphics[width=0.7\textwidth]{images/chapitre 2 - contexte/organigrame.png}
    \caption{Organigramme de Marketing Confort}
    \label{fig:organigramme_mc}
\end{figure}

\subsection{Compétences techniques}
\label{subsec:competences-techniques}

Marketing Confort dispose de compétences techniques diversifiées, couvrant :

\begin{itemize}
    \item Développement Web \& Mobile
    \item Cloud Computing
    \item Marketing Digital \& Production Audiovisuelle
    \item Coaching Agile \& DevOps
    \item Intelligence Artificielle
\end{itemize}

\begin{figure}[H]
    \centering
    \includegraphics[width=0.8\textwidth]{images/chapitre 2 - contexte/competences.png}
    \caption{Compétences de l'entreprise}
    \label{fig:competences_mc}
\end{figure}

\section{Présentation du projet MobilLoca}
\label{sec:presentation-projet}

\subsection{Contexte du projet}
\label{subsec:contexte-projet}

Le projet \textbf{MobilLoca} a été initié à la demande d'\textbf{Adanev}, une entreprise française spécialisée dans le transport de personnes à mobilité réduite. Bien qu'Adanev dispose déjà d'une application de transport avec chauffeur (\textit{Adanev Cab}), elle souhaitait étendre son offre en proposant un service de location de véhicules sans chauffeur, avec une gestion centralisée et numérique de l'ensemble du processus.

\subsection{Problématique}
\label{subsec:problematique}

La gestion manuelle des réservations, de la flotte, des documents administratifs et des paiements entraînait plusieurs limitations :

\begin{itemize}
    \item Manque de visibilité sur les performances globales
    \item Processus longs et sujets aux erreurs
    \item Expérience client peu fluide et manque d'autonomie
    \item Difficulté à suivre les maintenances et les documents expirés
\end{itemize}

\subsection{Objectifs du projet}
\label{subsec:objectifs-projet}

\textbf{MobilLoca} vise à :

\begin{itemize}
    \item \textbf{Automatiser} la gestion des réservations, contrats, paiements et documents
    \item \textbf{Centraliser} la gestion de la flotte et des agences
    \item \textbf{Offrir une expérience client intuitive} via une application mobile
    \item \textbf{Fournir des outils d'aide à la décision} via des tableaux de bord et rapports analytiques
    \item \textbf{Garantir une solution scalable et sécurisée}
\end{itemize}

\subsection{Solution proposée}
\label{subsec:solution-proposee}

La solution retenue est une plateforme digitale complète comprenant :

\begin{itemize}
    \item Un \textbf{Back-Office Web} pour les agents : gestion des véhicules, réservations, documents, paiements, statistiques.
    \item Une \textbf{Application Mobile} pour les clients : consultation du catalogue, réservation en ligne, paiement sécurisé, gestion des documents et historique.
\end{itemize}

\section{Planification du projet}
\label{sec:planification-projet}

\subsection{Méthodologie de gestion de projet}
\label{subsec:methodologie-gestion}

Une approche \textbf{Agile Scrum} a été adoptée pour garantir flexibilité, réactivité et livraison incrémentale. Le projet a été découpé en sprints de 2 à 3 semaines, avec des revues régulières et des ajustements continus.

\subsection{Phases principales}
\label{subsec:phases-principales}

\begin{enumerate}
    \item \textbf{Spécifications et conception} (Février 03 - Mai 01 2025)
    \item \textbf{Développement Front-End Web} (Mars 03 - Mai 16 2025)
    \item \textbf{Développement Front-End Mobile} (Février 16 - Mai 16 2025)
    \item \textbf{Structuration} (Avril 17 - Mai 17 2025)
    \item \textbf{Initiation des micro-services Back-End} (Mai 19 - Juin 02 2025)
    \item \textbf{Développement des micro-services Back-End} (Mai 19 - Septembre 02 2025)
    \item \textbf{Intégration des micro-service dans les applications} (Juilled 03 - Septembre 02 2025)
\end{enumerate}

\begin{figure}[H]
    \centering
    \includegraphics[width=\textwidth]{images/chapitre 2 - contexte/gant.png}
    \caption{Diagramme de Gantt du projet MobilLoca}
    \label{fig:gantt_mobilloca}
\end{figure}
\begin{figure}[H]
    \centering
    \includegraphics[width=\textwidth]{images/chapitre 2 - contexte/mobiloca_timeline.png}
    \caption{Diagramme de Gantt du projet MobilLoca - Derniers phases }
    \label{fig:gantt_mobilloca}
\end{figure}

\subsection{Outils de collaboration}
\subsubsection{Les outils :}
\label{subsec:outils-collaboration}

Pour garantir une communication efficace et une gestion de projet optimale, nous avons utilisé plusieurs outils collaboratifs adaptés à nos besoins. Chaque outil joue un rôle spécifique dans le processus de développement, permettant à l'équipe de rester synchronisée et productive.

\vspace{0.5em} % petit espace avant la liste

% Jira
\noindent
\begin{minipage}[H]{0.1\textwidth}
    \includegraphics[width=\linewidth]{images/logos/jira.png}
\end{minipage}%
\hfill
\begin{minipage}[H]{0.85\textwidth}
    \textbf{Jira} : Outil de gestion des tâches et des sprints, Jira nous a permis de planifier, suivre et prioriser les activités de développement selon la méthodologie Agile Scrum. Chaque tâche, user story ou bug était clairement assigné, ce qui garantissait transparence et traçabilité tout au long du projet.
\end{minipage}

\vspace{0.5em} % espace entre les outils

% GitLab
\noindent
\begin{minipage}[H]{0.1\textwidth}
    \includegraphics[width=\linewidth]{images/logos/gitlab.png}
\end{minipage}%
\hfill
\begin{minipage}[H]{0.85\textwidth}
    \textbf{GitLab} : Nous avons utilisé GitLab pour le versioning et l'intégration continue. Chaque membre de l'équipe pouvait gérer son code via des branches, des commits et des merge requests. Les pipelines CI/CD automatisés ont permis de tester et déployer les applications sans erreurs, améliorant la qualité globale du projet.
\end{minipage}

\vspace{0.5em}

% Discord
\noindent
\begin{minipage}[H]{0.1\textwidth}
    \includegraphics[width=\linewidth]{images/logos/discord.png}
\end{minipage}%
\hfill
\begin{minipage}[H]{0.85\textwidth}
    \textbf{Discord} : Pour la communication en temps réel, nous avons utilisé Discord, ce qui a facilité les discussions instantanées, les réunions d'équipe virtuelles et le partage de fichiers. L'outil a contribué à maintenir un flux de communication constant entre tous les membres.
\end{minipage}

\vspace{0.5em}

% Titan-Mail
\noindent
\begin{minipage}[H]{0.1\textwidth}
    \includegraphics[width=\linewidth]{images/logos/titan.png}
\end{minipage}%
\hfill
\begin{minipage}[H]{0.85\textwidth}
    \textbf{Titan-Mail} : Titan-Mail a été utilisé pour la messagerie professionnelle, garantissant la gestion formelle des communications, le suivi des échanges et la réception des notifications importantes liées au projet.
\end{minipage}


\subsubsection{Extraits et exemples :}

% === Jira examples ===
\noindent
\textbf{Jira :} Exemples de gestion des sprints et des tâches.

\begin{figure}[H]
    \centering
    \includegraphics[width=0.9\textwidth]{images/chapitre 2 - contexte/board.png}
    \caption{Exemple d’un sprint board dans Jira}
    \label{fig:jira-sprint-board}
\end{figure}

\vspace{1em}

% === GitLab examples ===
\noindent
\textbf{GitLab :} Exemples de gestion du code, merge requests et pipelines CI/CD.

\begin{figure}[H]
    \centering
    \includegraphics[width=0.85\textwidth]{images/chapitre 2 - contexte/merge-request.png}
    \caption{Exemple de merge request dans GitLab}
    \label{fig:gitlab-mr}
\end{figure}

\begin{figure}[H]
    \centering
    \includegraphics[width=0.85\textwidth]{images/chapitre 2 - contexte/git-lab-modules.png}
    \caption{Modules de projet et structure dans GitLab}
    \label{fig:gitlab-modules}
\end{figure}

\vspace{1em}

% === Discord examples ===
\noindent
\textbf{Discord :} Communication en temps réel et suivi des discussions de projet.

\begin{figure}[H]
    \centering
    \includegraphics[width=0.85\textwidth]{images/chapitre 2 - contexte/asking-devops-engineer-to-validate-mr.png}
    \caption{Exemple de demandes et discussions dans Discord}
    \label{fig:discord-requests}
\end{figure}

\vspace{1em}


\section{Conclusion}
\label{sec:conclusion-contexte}

Ce chapitre a permis de situer le projet \textbf{MobilLoca} dans son contexte organisationnel et stratégique. Nous avons présenté l'entreprise d'accueil, ses services, sa vision, ainsi que les enjeux et objectifs du projet. La méthodologie Agile et les outils de gestion mis en place assurent un cadre de travail structuré et réactif. Le chapitre suivant détaillera l'analyse des besoins et la conception technique de l'application.