
\chapter{Introduction}
\section*{Introduction Générale}

Ce rapport de \textbf{Projet de Fin d'Année (PFA)} présente le développement du projet \textbf{AutoStory}, réalisé dans le cadre de la formation en \textbf{Ingénierie Logicielle et Intelligence Artificielle} à l'\textbf{École Marocaine des Sciences de l'Ingénieur (EMSI)}.

\subsection*{Contexte}

Le marketing digital et le secteur automobile connaissent actuellement une transformation profonde portée par l'essor des technologies d'\textbf{intelligence artificielle générative}. Les entreprises du secteur automobile ont un besoin croissant de \textbf{contenus publicitaires personnalisés et produits rapidement} pour promouvoir leurs véhicules sur les plateformes digitales. Cependant, la production vidéo traditionnelle reste \textbf{coûteuse, chronophage et nécessite des compétences spécialisées} (scénaristes, voix-off professionnelles, monteurs vidéo).

L'émergence des \textbf{Large Language Models (LLMs)} tels que GPT-4, Claude et Google Gemini ouvre de nouvelles perspectives pour l'automatisation de la création de contenu narratif. Parallèlement, les technologies de \textbf{synthèse vocale (Text-to-Speech)} et de \textbf{composition vidéo programmatique} permettent aujourd'hui de produire des supports publicitaires de qualité sans intervention humaine extensive.

\subsection*{Problématique}

Face à ce contexte, plusieurs questions se posent :

\begin{itemize}
    \item Comment exploiter les capacités des LLMs pour générer des scripts publicitaires techniques et cohérents pour des véhicules automobiles ?
    \item Comment automatiser l'ensemble de la chaîne de production vidéo, de la génération du texte à la composition finale ?
    \item Comment garantir la qualité narrative et la cohérence des contenus générés automatiquement ?
    \item Quelle architecture logicielle adopter pour assurer modularité, scalabilité et maintenabilité du système ?
\end{itemize}

\subsection*{Objectifs du projet}

Le projet \textbf{AutoStory} vise à répondre à ces problématiques en développant un système complet de génération automatisée de vidéos publicitaires pour le secteur automobile. Les objectifs spécifiques sont les suivants :

\begin{enumerate}
    \item Mettre en place un système de génération de scripts narratifs techniques utilisant \textbf{Google Gemini 2.5 Flash}
    \item Développer un module de synthèse vocale en français avec \textbf{gTTS (Google Text-to-Speech)}
    \item Implémenter un système de composition vidéo automatisé avec \textbf{MoviePy}
    \item Créer un module de classification automatique des styles de véhicules (sportif, luxe, familial, écologique)
    \item Concevoir une interface utilisateur moderne avec \textbf{React et TypeScript}
    \item Adopter une architecture logicielle \textbf{modulaire et scalable} facilitant l'évolution future du système
\end{enumerate}

\subsection*{Démarche et structure du rapport}

Ce travail s'inscrit dans une démarche pédagogique rigoureuse, combinant apprentissage théorique et application pratique. Il nous permet de mobiliser nos connaissances en \textbf{intelligence artificielle}, \textbf{développement logiciel}, \textbf{traitement du langage naturel} et \textbf{architecture applicative}.

Le présent rapport est structuré de la manière suivante :

\begin{itemize}
    \item Le \textbf{Chapitre 1} présente le cahier des charges du projet AutoStory
    \item Le \textbf{Chapitre 2} expose le contexte général, l'état de l'art des technologies IA génératives et le positionnement d'AutoStory
    \item Le \textbf{Chapitre 3} détaille l'analyse et la spécification des besoins fonctionnels et non-fonctionnels
    \item Le \textbf{Chapitre 4} décrit la conception de l'architecture et les choix techniques
    \item Le \textbf{Chapitre 5} présente la réalisation et la mise en œuvre du système
    \item Le \textbf{Chapitre 6} conclut par un bilan du projet et présente les perspectives d'évolution
\end{itemize}
